\documentclass[USenglish, aspectratio = 169]{beamer}

\usetheme{SFUBeamer}
\usepackage{style}

% OTHER PACKAGES-------------------------------
\usepackage{caption}
\usepackage{booktabs}

%%%%%%%%%%%%%%%%%%%%%%%%%%%%%%%%%%%%%%%%%%%%%%%%%%%
%
%  DEFINITIONS 
%
%%%%%%%%%%%%%%%%%%%%%%%%%%%%%%%%%%%%%%%%%%%%%%%%%%%

\newcommand{\norm}[1]{{\left\| \, #1 \, \right\|}}

%%%%%%%%%%%%%%%%%%%%%%%%%%%%%%%%%%%%%%%%%%%%%%%%%%%
%
%  MAIN INFORMATION
%  
%  Picture in title frame is, for now, hard-coded in 
%  beamerinnerthemeSFUBeamer.sty.
%
%%%%%%%%%%%%%%%%%%%%%%%%%%%%%%%%%%%%%%%%%%%%%%%%%%%

\title[An SFU Beamer Template]{SFUBeamer: a PDFLatex-based beamer template for the SFU community}
\subtitle{A spin-off of the OsloMet template}

\author[J. Almonacid]{Javier Almonacid}
\institute{Department of Mathematics | Simon Fraser University} 
\date{July 14, 2023}  % Can also be used as date


\begin{document}

\section{Overview}
% Use
%
%     \begin{frame}[allowframebreaks]
%
% if the TOC does not fit one frame.
\begin{frame}{Table of contents}
    \tableofcontents
\end{frame}


\section{Mathematics}
\subsection{Theorem}

%% Disable the logo in the lower right corner:
\hidelogo

\begin{frame}{Mathematics}

    \begin{theorem}[Fermat's little theorem]
        For a prime~\(p\) and \(a \in \mathbb{Z}\) it holds that \(a^p \equiv a \pmod{p}\).
    \end{theorem}

    \begin{proof}
        The invertible elements in a field form a group under multiplication.
        In particular, the elements
        \begin{equation*}
            1, 2, \ldots, p - 1 \in \mathbb{Z}_p
        \end{equation*}
        form a group under multiplication modulo~\(p\).
        This is a group of order \(p - 1\).
        For \(a \in \mathbb{Z}_p\) and \(a \neq 0\) we thus get \(a^{p-1} = 1 \in \mathbb{Z}_p\).
        The claim follows.
    \end{proof}
\end{frame}

%% Enable the logo in the lower right corner:
\showlogo

\subsection{Example}

\begin{frame}{Mathematics}

    \begin{example}
        The function \(\phi \colon \mathbb{R} \to \mathbb{R}\) given by \(\phi(x) = 2x\) is continuous at the point \(x = \alpha\),
        because if \(\epsilon > 0\) and \(x \in \mathbb{R}\) is such that \(\lvert x - \alpha \rvert < \delta = \frac{\epsilon}{2}\),
        then
        \begin{equation*}
            \lvert \phi(x) - \phi(\alpha)\rvert = 2\lvert x - \alpha \rvert < 2\delta = \epsilon.
        \end{equation*}
    \end{example}
\end{frame}

\section{Highlighting}
\SectionPage{0.6\textwidth}

\begin{frame}{Highlighting}

    Sometimes it is useful to \alert{highlight} certain words in the text.

    \begin{alertblock}{Important message}
        If a lot of text should be \alert{highlighted}, it is a good idea to put it in a box.
    \end{alertblock}

    You can also highlight with the \structure{structure} colour.
\end{frame}

\section{Lists}

\begin{frame}{Lists}

    \begin{itemize}
        \item
        Bullet lists are marked with a dark red box.
    \end{itemize}

    \begin{enumerate}
        \item
        \label{enum:item}
        Numbered lists are marked with a black number inside a dark red box.
    \end{enumerate}

    \begin{description}
        \item[Description] highlights important words with gray text.
    \end{description}

    Items in numbered lists like \enumref{enum:item} can be referenced with a dark red box.

    \begin{example}
        \begin{itemize}
            \item
            Lists change colour after the environment.
        \end{itemize}
    \end{example}
\end{frame}

\section{Effects (longer titles work best)}
%---------------------------------------------------------------------
% \SectionPageUpNext is a feature that requires further work. 
% If you are not comfortable with the result, use \SectionPage{width}
% instead.
%---------------------------------------------------------------------
\SectionPageUpNext{0.6\textwidth}{ \color{SFUDarkGray}
This is a good point to remind your audience what they have seen so far.
    \begin{itemize}
    \item \color{SFUDarkGray} Item 1.
    \item \color{SFUDarkGray} Item 2.
    \item \color{SFUDarkGray} Item 3.
    \end{itemize}
}

\begin{frame}{Effects}
    \begin{columns}[onlytextwidth]
        \begin{column}{0.49\textwidth}
            \begin{enumerate}[<+-|alert@+>]
                \item
                Effects that control

                \item
                when text is displayed

                \item
                are specified with <> and a list of slides.
            \end{enumerate}

            \begin{theorem}<2>
                This theorem is only visible on slide number 2.
            \end{theorem}
        \end{column}
        \begin{column}{0.49\textwidth}
            Use \textbf<2->{textblock} for arbitrary placement of objects.

            \pause
            \medskip

            It creates a box
            with the specified width (here in a percentage of the slide's width)
            and upper left corner at the specified coordinate (x, y)
            (here x is a percentage of width and y a percentage of height).
        \end{column}
    \end{columns}
    
    \only<1, 3>
    {
        \begin{textblock}{0.3}(0.45, 0.55)
            \includegraphics[width = \textwidth]{example-image-a}
        \end{textblock}
    }
\end{frame}

\section{References}

\begin{frame}[shrink=10]{References}

    \begin{thebibliography}{}     
        
        \bibitem{b4}{
		Y. Colin de V\`erdiere. \normalfont
		Spectral theory of pseudo-differential operators of degree 0 and application to forced linear waves.
		\emph{Anal. PDE} 13 (2020), no. 5, 1521–1537}
		
		\bibitem{b2}{
        Y. Colin de V\`erdiere \& L. Saint-Raymond. \normalfont
		Attractors for two dimensional waves with homogeneous Hamiltonians of degree 0.
        \emph{Commun. Pure Appl. Anal.} 73 (2020), no. 2, 421--462.}
		
		\bibitem{b6}{
		G. Davis, T. Jamin, J. Deleuze, S. Joubaud \& T. Dauxois. \normalfont
		Succession of resonances to achieve internal wave turbulence.
		Phys. Rev. Lett 124 (2020), 204502.}	
		
		\bibitem{b3}{
        S. Dyatlov \& M. Zworski. \normalfont
        Microlocal analysis of forced waves.
        \emph{Pure Appl. Anal.} 1 (2019), 359--384.}	
        
        \bibitem{b5}{
		J. Galkowski \& M. Zworski. \normalfont
		Viscosity limits for 0th order pseudo-differential operators.
		arXiv:1912.09840 (December 2019).}
		
		\bibitem{b1}{
		L.R.M. Maas. \normalfont
		Wave attractors: linear yet nonlinear.
		\emph{Int. J. Bifurcat. Chaos} 15 (2005), no. 9, 2757--2782.}

    \end{thebibliography}

	\begin{center}
	\textbf{Thank you!}
	\end{center}
\end{frame}




\end{document}
